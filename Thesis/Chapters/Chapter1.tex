% Chapter Template

\chapter{Introduction} % Main chapter title

\label{Chapter1} % Change X to a consecutive number; for referencing this chapter elsewhere, use \ref{ChapterX}

As humanity strives to better understand the world and universe around it, the physical limitations of our species become more and more apparent. While we have made considerable progress in our struggles to move forward, such as being able to record the movement of a light particle on camera despite it being the fastest moving object we know of so far\textsuperscript{\cite{velten2013femto}}, or being able to capture an image of a black hole\textsuperscript{\cite{landau_2019}}, such achievements would not have been possible if our scientists did not have realistic expectations of how they should approach these challenges, or the expected results. 

In order to achieve what they have, scientists needed to first understand the phenomena they were studying: the light having the particular properties of both particle and wave and the ability of black holes to distort space around them. All of this would not have been possible without simulations. 

%----------------------------------------------------------------------------------------
%	SECTION 1
%----------------------------------------------------------------------------------------

\section{Electromagnetic Simulations}
With the fast development of technology came new opportunities for gaining a better understanding of vast natural phenomena. We will be focusing on one of them, that being Electromagnetic Wave Propagation. 

As one can imagine, analyzing electromagnetic fields through plain observation is near impossible, with only a few exceptions\textsuperscript{\cite{cao2005first}}. Even if we supposed that it was possible to easily achieve an acceptable amount of information from observing experiments, the cost and quality of the resulting data would mostly be of scientific use, with little to no practical use whatsoever. Considering that electromagnetic waves are widely used in almost every industry, either as part of the building process or as a finished product, having data that cannot be used practically does not help. 

That is why, thanks to the progress made in the computational capabilities of computers so far and the use of the theories and formulas gathered from past scientific endeavors, one can create data that is a close approximate of reality. Both shall be discussed in this chapter, but we cannot proceed without first going into what is believed by scientists to be the equations that govern large-scale electromagnetic phenomena\textsuperscript{\cite{stratton2007electromagnetic}}: Maxwell Equations.

%-----------------------------------
%	SUBSECTION 1
%-----------------------------------
\subsection{Maxwell Equations}

As mentioned above, Maxwell Equations are believed to dictate the behavior of all kinds of electromagnetic fields at a macroscopic level. These equations are the following:

\begin{equation}
	\label{eqn:electricinduction}
	\vec{\nabla} \times \vec{E}(\vec{r},t) = - \frac{\partial \vec{B}(\vec{r},t)}{\partial t}
\end{equation}
\begin{equation}
	\label{eqn:amperesLaw}
	\vec{\nabla} \times \vec{H}(\vec{r},t) = \vec{J}(\vec{r},t) + \frac{\partial \vec{D}(\vec{r},t)}{\partial t}
\end{equation}
\begin{equation}
	\label{eqn:magneticDivergence}
	\vec{\nabla} \cdot \vec{B}(\vec{r},t) = 0
\end{equation}
\begin{equation}
	\label{eqn:gausslaw}
	\vec{\nabla} \cdot \vec{D}(\vec{r},t) = \rho (\vec{r})
\end{equation}

To give a brief explanation over the meaning of each equation:

Equation \ref{eqn:electricinduction} explains the effects of the electric field $\vec{E}$ on the rate of change of the magnetic induction $\vec{B}$. This can also be referred to as the equation of electromagnetic induction, where the right hand side is the EMF or voltage and the left hand side is the magnetic flux. To those who study this particular area of physics, this equation will seem familiar, because it was derived from Faraday's law of induction.

\begin{figure}
	\centering
	\includegraphics{Figures/faradayexp}
	\decoRule
	\caption[Faraday's Experiment]{Faraday's Experiment,\textsuperscript{\cite{poyser1918magnetism}}} which resulted in the law that was later used by Maxwell in making his equation.
	\label{fig:faradayexp}
\end{figure}

Equation \ref{eqn:amperesLaw} is also known as Ampère–Maxwell law, because although it originated from Ampère, the current form was derived by Maxwell to include the magnetic current density $\vec{J}$. The equation explains the effects of the magnetic field on the electric current.

Equation \ref{eqn:magneticDivergence}, otherwise known as Gauss's law for magnetism, talks about the divergence of the magnetic field. According to this law the divergence is always 0. What this means is that in a magnetic field there is no such thing as a source (positive divergence) or a sink (negative divergence), rather a magnetic field can be more closely compared to a closed loop that flows in one direction. That is why every magnet that we know of has two poles. To translate this into something more easily understandable, it basically means that, if we were to pick any subset of the area of a magnetic field, no matter what area we pick, we would have vector fields going inside this area, and vector fields going outside in equal number. A simple representation of this rule can be seen in \ref{fig:magneticdivergence}, where it can be noted that the number of vectors heading towards the north pole are equal to the vectors going outside of it. Interestingly enough, if the bar magnet were to be cut in half, then the result would be two smaller bar magnets with 2 poles each and the exact same vector field.

\begin{figure}
	\centering
	\includegraphics[scale=0.9]{Figures/magneticdivergence}
	\decoRule
	\caption[Magnetic Divergence]{A crude representation of magnetic divergence through the use of a bar magnet.}
	\label{fig:magneticdivergence}
\end{figure}

Equation \ref{eqn:gausslaw}, is the main Gauss law for electric currents. It looks rather similar to his law of magnetism on the left hand side; the previous magnetic divergence is now replaced with the electric divergence. The bigger difference is the $\rho$ on the right hand side, which is the density of charge of the electric field. In basic terms, it means that the divergence of the electric field is equal to the charge density for that point.

For the above equations, we also have the following material relations which will be useful later on:
\begin{equation}
	 \vec{D}(\vec{r},t) = \epsilon(\vec{r}) \cdot \vec{E}(\vec{r},t)
\end{equation}
\begin{equation}
	\vec{B}(\vec{r},t) = \mu(\vec{r}) \cdot \vec{H}(\vec{r},t)
\end{equation}

The equations above are shown in their derivative form, but they can also be shown as their integral equivalent. There is no difference in implementation, regardless of which form is used.

%-----------------------------------
%	SUBSECTION 2
%-----------------------------------

\subsection{Solving the Wave Equation}
Using the formulas above, we can derive the electromagnetic wave equation, which is needed to proceed with the implementation further. Firstly for convenience, we will assume that the environment is the vacuum of space. Secondly, we will also assume that there is no charge in this space. This means that $\epsilon(\vec{r})=\epsilon_{0}$ and $\mu(\vec{r})=\mu_{0}$. This allows us to simplify the above equations and give us the numerical solution for the wave equation

\begin{equation}
	\label{eqn:waveEquation}
	\Delta\vec{E}(\vec{r},t) - \frac{1}{c^2} \frac{\partial^2 \vec{E}(\vec{r},t)}{\partial t^2} = \mu_{0} \frac{\partial \vec{J}(\vec{r},t)}{\partial t},
\end{equation}

where: $$c = \frac{1}{\sqrt{\mu_{0}\epsilon_{0}}}$$

This equation is useful because it will be used to derive the formulas that are going to be needed for the simulations. Going forwards, this will be used as a basis to adapt our solution to every environment, regardless of dimensionality. From this point, there are many ways to proceed. Some of the most notable are the Finite Element Method (FEM), the Finite Integration Technique (FIT), and the Finite Difference Time Domain Method (FDTD). All of them are also known as Approximation Methods.

FEM is a well known numerical method used to obtain an approximation for a given boundary value problem. The basic principle is to divide a system into smaller subsets, called finite elements (hence the name), which are much simpler to solve. This is done by discretizing the given space for each of its dimensions, and then constructing a mesh of the object. As a result, from the initial boundary value problem we get a system of equations that are further used to approximate each singular simplified function over the given domain. These equations are then compiled together into a system of equations that is then used to model the initial problem. The solution is then approximated by solving this system and minimizing the error function.\textsuperscript{\cite{logan2011first}}

The finite integration technique (FIT) is a bit more straightforward. It can help numerically solve electromagnetic field problems by disctretizing in both the time and frequency domain. The first to introduce this technique was Thomas Weiland in 1977.\textsuperscript{\cite{weiland1977discretization}} It has later seen continuous improvements and can now cover all electromagnetic problems and applications. This approach works by using the Maxwell equations above and applying their integral form to a set of staggered grids (e.g. Cartesian grid). This allows for a memory efficient implementation as well as giving the ability to handle different boundary conditions and variable material properties.

Finally we have a well known computational electromagnetic technique used for approximation, the Finite Difference Time Domain Method. It is arguably the easiest method out of the three to understand and implement, which is surprising when considering the capabilities it has in solving wave equations. This simplicity is also the reason it was chosen for this thesis, as it is the only technique that one person can realistically implement by themselves in a reasonable time frame.\textsuperscript{\cite{davidson2010computational}} As the name implies this is a time-domain method, meaning that a wide range of frequencies can be covered with a single simulation run. The only caveat is that the time step needs to be small enough to not cause any instabilities in the system.

%----------------------------------------------------------------------------------------
%	SECTION 2
%----------------------------------------------------------------------------------------

\section{Finite Difference Time Domain Method}

FDTD belongs in the general class of grid-based differential time-domain numerical modeling methods. Maxwell's equations (in partial differential form) are modified to central-difference equations, discretized, and implemented in software. The equations are solved in a cyclic manner: the electric field is solved at a given instant in time, then the magnetic field is solved at the next instant in time, and the process is repeated over and over again.

The basic FDTD algorithm traces back to a seminal 1966 paper by Kane Yee in IEEE Transactions on Antennas and Propagation. Allen Taflove originated the descriptor "Finite-difference time-domain" and its corresponding "FDTD" acronym in a 1980 paper in IEEE Trans. Electromagn. Compat.. Since about 1990, FDTD techniques have emerged as the primary means to model many scientific and engineering problems addressing electromagnetic wave interactions with material structures. An effective technique based on a time-domain finite-volume discretization procedure was introduced by Mohammadian et al. in 1991.[12] Current FDTD modeling applications range from near-DC (ultralow-frequency geophysics involving the entire Earth-ionosphere waveguide) through microwaves (radar signature technology, antennas, wireless communications devices, digital interconnects, biomedical imaging/treatment) to visible light (photonic crystals, nanoplasmonics, solitons, and biophotonics). Approximately 30 commercial and university-developed software suites are available.

\section{FDTD Implementation}

Sed ullamcorper quam eu nisl interdum at interdum enim egestas. Aliquam placerat justo sed lectus lobortis ut porta nisl porttitor. Vestibulum mi dolor, lacinia molestie gravida at, tempus vitae ligula. Donec eget quam sapien, in viverra eros. Donec pellentesque justo a massa fringilla non vestibulum metus vestibulum. Vestibulum in orci quis felis tempor lacinia. Vivamus ornare ultrices facilisis. Ut hendrerit volutpat vulputate. Morbi condimentum venenatis augue, id porta ipsum vulputate in. Curabitur luctus tempus justo. Vestibulum risus lectus, adipiscing nec condimentum quis, condimentum nec nisl. Aliquam dictum sagittis velit sed iaculis. Morbi tristique augue sit amet nulla pulvinar id facilisis ligula mollis. Nam elit libero, tincidunt ut aliquam at, molestie in quam. Aenean rhoncus vehicula hendrerit.

\subsection{Project Plan, Requirements, and Tools Used}

Sed ullamcorper quam eu nisl interdum at interdum enim egestas. Aliquam placerat justo sed lectus lobortis ut porta nisl porttitor. Vestibulum mi dolor, lacinia molestie gravida at, tempus vitae ligula. Donec eget quam sapien, in viverra eros. Donec pellentesque justo a massa fringilla non vestibulum metus vestibulum. Vestibulum in orci quis felis tempor lacinia. Vivamus ornare ultrices facilisis. Ut hendrerit volutpat vulputate. Morbi condimentum venenatis augue, id porta ipsum vulputate in. Curabitur luctus tempus justo. Vestibulum risus lectus, adipiscing nec condimentum quis, condimentum nec nisl. Aliquam dictum sagittis velit sed iaculis. Morbi tristique augue sit amet nulla pulvinar id facilisis ligula mollis. Nam elit libero, tincidunt ut aliquam at, molestie in quam. Aenean rhoncus vehicula hendrerit.


\subsection{Computational Limitations and Inaccuracies Explained}

Sed ullamcorper quam eu nisl interdum at interdum enim egestas. Aliquam placerat justo sed lectus lobortis ut porta nisl porttitor. Vestibulum mi dolor, lacinia molestie gravida at, tempus vitae ligula. Donec eget quam sapien, in viverra eros. Donec pellentesque justo a massa fringilla non vestibulum metus vestibulum. Vestibulum in orci quis felis tempor lacinia. Vivamus ornare ultrices facilisis. Ut hendrerit volutpat vulputate. Morbi condimentum venenatis augue, id porta ipsum vulputate in. Curabitur luctus tempus justo. Vestibulum risus lectus, adipiscing nec condimentum quis, condimentum nec nisl. Aliquam dictum sagittis velit sed iaculis. Morbi tristique augue sit amet nulla pulvinar id facilisis ligula mollis. Nam elit libero, tincidunt ut aliquam at, molestie in quam. Aenean rhoncus vehicula hendrerit.